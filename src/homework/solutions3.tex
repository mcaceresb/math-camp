%----------------------------------------------------------------------
\documentclass{article}

\usepackage[homework]{brownpreamble}
\lhead{\color{light-gray} \itshape Math Camp 2021}
\rhead{\color{light-gray} \itshape Suggested Solutions 3}
\renewcommand\sectiontype{Suggested Solutions \thesection\ }
% \renewcommand\thesection{\arabic{section}}
\setcounter{section}{2}

%----------------------------------------------------------------------
\begin{document}
\displayoptions

% ---------------------------------------------------------------------
\section{}

\begin{enumerate}[1.]
  \item {\itshape
    Take a collection of functions with $f_i: \Omega \to \mathbb{R}^N$, $\Omega \subseteq \mathbb{R}^M, i \in \mathbb{N}$. The collection $\set{f_i}$ defines a sequence of funtions, and for each $x \in \Omega$ we have a possibly different sequence $\set{f_i(x)}$ in $\mathbb{R}^N$.

    Let $\set{f_i}$ be a sequence of functions, with $f_i: \Omega \to \mathbb{R}^N$ and $\Omega \subseteq \mathbb{R}^M$. We say that $\set{f_i}$ \textbf{point-wise convrges} to $f: \Omega_0 \to \mathbb{R}^N$ if $x \in \Omega_0 \implies f_i(x) \to f(x)$.

    Let $\set{f_i}$ be a sequence of functions, with $f_i: \Omega \to \mathbb{R}^N$ and $\Omega \subseteq \mathbb{R}^M$. We say that $\set{f_i}$ \textbf{uniformly convrges} to $f: \Omega_0 \to \mathbb{R}^N$ if $\forall \varepsilon > 0 ~~ \exists I_0(\varepsilon)$ s.t. for $i > I_0(\varepsilon)$ we have $\Fnorm{f_i(x) - f(x)} < \varepsilon$. }

    \begin{enumerate}[a)]
      \item \textit{Let $f_i(x) = x / i$ and $f(x) = 0$. Check that $f_i \to f$ point-wise.}

        \solution Pick an arbitrary $x$ and $\varepsilon > 0$. We have the sequence $x_m = x / m$, and for $N > \norm{x} / \varepsilon$ we have that
        \[
          \norm{x / m} < \norm{x} / N < \varepsilon
        \]

        whenever $m > N$. Hence for any given $x$, $f_i(x) \to f(x)$.

      \item \textit{Show $f_i$ defined above does not converge uniformly to $f$.}

        \solution Crucially, the number $I_0(\varepsilon)$ we choose in this case cannot depend on $x$, only on $\varepsilon$. Suppose that it does converge uniformly, then for any $\varepsilon > 0$ we find $I_0(\varepsilon)$ s.t. $i > I_0(\varepsilon)$ gives
        \[
          \Fnorm{f_i(x) - f(x)} < \varepsilon
          \iff
          \dfrac{1}{i} \Fnorm{x} < \varepsilon
          \iff
          \Fnorm{x} < \varepsilon i
        \]

        However, for fixed $\varepsilon$ and $I_0(\varepsilon)$, take any given $i > I_0(\varepsilon)$, for instance $i = I_0(\varepsilon) + 1$, and $x = \sqrt{M} \varepsilon i + \mathbf{1}$ (with $\mathbf{1}$ a vector of ones and $M$ the dimension of $\mathbb{R}^M$, since we said $\Omega \subseteq \mathbb{R}^M$) will violate the equation above, a contradiction.

      \item \textit{Show uniform convergence implies point-wise convergence.}

        \solution This follows directly from the definition. Pick an $x^\prime \in \Omega_0$; $f_i \to f$ point-wise if the sequence $x^\prime_i = f_i(x^\prime) \to f(x)$. This is true if for every $\varepsilon > 0$ there exists some $I$ s.t. $i > I$ gives
        \[
          \Fnorm{f_i(x^\prime) - f(x^\prime)} < \varepsilon
        \]

        If $f_i \to f$ uniformly, then we know that $I_0(\varepsilon)$ exists s.t.
        \[
          \Fnorm{f_i(x) - f(x)} < \varepsilon
        \]

        for every $x \in \Omega_0$, which means it must also be true of $x^\prime$. Let $I = I_0(\varepsilon)$ and we are done.
    \end{enumerate}

  \item {\itshape Let $A \subseteq \mathbb{R}^N$ be a convex set. $f: A \to \mathbb{R}^N$ is quasiconcave if for any $x, y \in A$ and $\alpha \in [0, 1]$ we have
      \begin{align*}
        f(\alpha x + (1 - \alpha) y) \ge \min\set{f(x), f(y)}
      \end{align*}

    and strictly quasiconcave if the above holds strictly. Show if $f$ is quasiconcave then $\argmax_{x \in A} f(x)$ is a convex set (recall the empty set is convex by vacuity). Further show that if $f$ is strictly quasiconcave then $\argmax_{x \in A} f(x)$ is a singleton or empty.}

    \solution We have that for any pair $x, y \in A$, and $\alpha \in [0, 1]$
      \[
        f(\alpha x + (1 - \alpha) y) \ge \min\set{f(x), f(y)}
      \]

      Furthermore,
      \[
        \argmax_{x \in A} f(x) = \set{x \in A: f(x) \ge f(y) ~~ \forall y \in A}
      \]

      We want to show that for any two points, $x^*, y^* \in \argmax_{x \in A} f(x)$, the following holds:
      \[
        \alpha x^* + (1 - \alpha) y^* \in \argmax_{x \in A} f(x)
      \]

      Note that since $A$ is convex and $x^*, y^* \in A$, we have $\alpha x^* + (1 - \alpha) y^* \in A$.

      If there is no maximum the above is true by vacuity: For any two points in the empty set, their combination is also in the empty set. This statement is true by vacuity, but perhaps the contrapositive makes it more clear: If the convex combination of any two points is not on the empty set for some $\alpha \in [0, 1]$ (true because there are no points in the empty set), then the two points are not in the set (also true because there are no points in the empty set).

      So let us consider the non-trivial case. $x^*, y^* \in \argmax$ gives that $f(x^*) \ge f(x) ~~ \forall x \in A$ and $f(y^*) \ge f(x) ~~ \forall x \in A$. Since $f$ is quasiconcave,
      \[
        f(\alpha x^* + (1 - \alpha) y^*) \ge \min\set{f(x^*), f(y^*)} \ge f(x) ~~ \forall x \in A
      \]

      which is exactly what we wanted to show, so $\argmax$ is a convex set. If we have \textit{strict} quasiconcavity and the maximum does not exist, then the $\argmax$ is empty. So, again, let us consider the non-trivial case: Suppose that $\argmax$ is not a singleton. We know that it must be convex, so for $x^*, y^* \in \argmax$ with $x^* \ne y^*$ we have
      \[
        f(\alpha x^* + (1 - \alpha) y^*) \in \argmax_{x \in A} f(x)
      \]

      However, by strict quasiconcavity we also have
      \[
        f(\alpha x^* + (1 - \alpha) y^*) > \min\set{f(x^*), f(y^*)} \ge f(x) ~~ \forall x \in A
      \]

      Since $A$ is convex and $x^*, y^* \in A$, we have that $\alpha x^* + (1 - \alpha) y^* \in A$, and thus
      \[
        f(\alpha x^* + (1 - \alpha) y^*) > f(\alpha x^* + (1 - \alpha) y^*)
      \]

      a contradiction. So if the $\argmax$ is not empty, it is a singleton.

  \item {\itshape
      Consider a continuous function $f: \mathbb{R}^N \to \mathbb{R}$. Show
    }

    \begin{enumerate}[a)]
      \item \textit{If $f$ is differentiable and $x^* \in \mathbb{R}^N$ is a local maximizer of $f$ then $\nabla f(x^*) = 0$.}

        \solution Suppose $x^*$ a local maximizer of $f$. Then $\exists \varepsilon > 0$ s.t.
        \[
          x \in B_{\varepsilon}(x^*)
          \implies
          f(x^*) \ge f(x)
        \]

        Since $f$ is differentiable, we know that $\dfrac{\partial}{x_j} f(x^*)$ exists for each $j$. Hence
        \[
          \dfrac{\partial}{\partial x_j} f(x^*)
          =
          \lim_{t \to 0}  \dfrac{f(x^* + t e_j) - f(x^*)}{t}
          =
          \lim_{t \to 0+} \dfrac{f(x^* + t e_j) - f(x^*)}{t}
          =
          \lim_{t \to 0-} \dfrac{f(x^* + t e_j) - f(x^*)}{t}
        \]

        where $\set{e_j}$ are the standard basis for $\mathbb{R}^N$. Further, for $t < \varepsilon$, we have that $\Fnorm{x^* + t e_j - x^*} = t < \varepsilon$. Hence $x^* + t e_j \in B_{\varepsilon}(x)$ and
        \begin{equation}
          \begin{array}{c}
          f(x^* + t e_j) \le f(x^*)
          \implies
          f(x^* + t e_j) - f(x^*)
          \le 0
          \\[6pt]
          t \ge 0
          \implies \dfrac{f(x^* + t e_j) - f(x^*)}{t}
          \le 0
          \\[6pt]
          t \le 0
          \implies \dfrac{f(x^* + t e_j) - f(x^*)}{t}
          \ge 0
          \end{array}
          \label{eq:ps3_eq1}
        \end{equation}

        This means that as $t \to 0^+$ or $t \to 0^-$ we get
        \begin{equation}
          \begin{array}{rl}
            \dfrac{\partial}{\partial x_j} f(x^*)
            & =
            \lim_{t \to 0+} \dfrac{f(x^* + t e_j) - f(x^*)}{t}
            \le 0
            \\[6pt]
            \quad\quad
            \dfrac{\partial}{\partial x_j} f(x^*)
            & =
            \lim_{t \to 0-} \dfrac{f(x^* + t e_j) - f(x^*)}{t}
            \ge 0
            \\[6pt]
            \implies
            \dfrac{\partial}{\partial x_j} f(x^*)
            & = 0
          \end{array}
          \nonumber
        \end{equation}

      \item \textit{If $f$ is twice continuously differentiable and $x^* \in \mathbb{R}^N$ is s.t. $\nabla f(x^*) = 0$, then if $x^*$ is a local maximizer the symmetric $N \times N$ Hessian $D^2f(x^*)$ is negative semidefinite. Extra credit: If $D^2f(x^*)$ is negative definite then $x^*$ is a strict local maximizer. (Hint: I used a Taylor expansion without the explicit remainder formula. For the extra-credit, I additionally leveraged the fact a matrix is ND iff it has all strictly negative eigenvalues, but there may be a way to do it without that.)}

        \solution Pick any vector $v$ and note $x^* + \varepsilon \cdot v / \Fnorm{v}$ is in the $\varepsilon$-neighborhood of $x^*$. Alternatively, any vector $z \in B_{\varepsilon}(x^*)$ can be expressed in this way. Now consider for a given $z$ the 2nd-order Taylor expansion of $g(\varepsilon) = f(x^* + \varepsilon v / \Fnorm{v})$ around $\varepsilon = 0$.
        \begin{align*}
          g(\varepsilon)
          =
          g(0)
          + g^\prime(0) \varepsilon
          + \dfrac{1}{2} g^{\prime\prime}(0) \varepsilon^2
          + h(\varepsilon) \varepsilon^2
        \end{align*}

        for some $h(\varepsilon) \to 0$ as $\varepsilon \to 0$. By the chain rule,
        \begin{align*}
          g^{\prime}(0)
          =
          \dfrac{v^T}{\Fnorm{v}} D_x f(x^*)
          \quad\text{and}\quad
          g^{\prime\prime}(0)
          =
          \dfrac{v^T}{\Fnorm{v}} D_x f(x^*) \dfrac{v}{\Fnorm{v}}
        \end{align*}

        However, from the FOC we know that $D_x f(x^*) = 0$.  Therefore we find
        \begin{align*}
          f(z) - f(x^*)
          &
          =
          \dfrac{1}{2}
          \dfrac{v^T}{\Fnorm{v}} D_x f(x^*) \dfrac{v}{\Fnorm{v}} \varepsilon^2
          +
          h(\varepsilon) \varepsilon^2
        \end{align*}

        for any $z \in B_{\varepsilon}(x)$.
        \begin{enumerate}[i)]
          \item For the first portion of the problem, we want to show that for an arbitrary vector $v$ we have $v^T D_x f(x^*) v \le 0$ given $x^*$ is a local maximizer. A local maximizer means there is some neighborhood $\delta$ s.t. $z \in B_{\delta}(x^*) \implies f(z) \le f(x^*)$. Fix an arbitrary non-zero vector $v$ and for $\varepsilon < \delta$ we have $z = x^* + \varepsilon v / \Fnorm{v} \in B_{\varepsilon}(x^*) \subset B_{\delta}(x^*)$. Therefore for arbitrary $v$ and small-enough $\varepsilon$ we find
            \begin{align*}
              0
              \ge
              &
              f(z) - f(x^*)
              =
              \dfrac{1}{2}
              \dfrac{v^T}{\Fnorm{v}} D_x f(x^*) \dfrac{v}{\Fnorm{v}} \varepsilon^2
              +
              h(\varepsilon) \varepsilon^2
              \\
              0
              \ge
              &
              v^T D_x f(x^*) v
              +
              2 \Fnorm{v}^2 h(\varepsilon)
            \end{align*}

            where the inequalities did not flip when we multiplied by $2, \Fnorm{v}^2, 1/\varepsilon^2$ as these are all $> 0$. Finally take $\varepsilon \to 0$ so $2 \Fnorm{v}^2 h(\varepsilon) \to 0$. Since $v$ is fixed the first term is constant and we have
            \begin{align*}
              0
              \ge
              v^T D_x f(x^*) v
            \end{align*}

            (i.e. sums preserve limits), which is what we wanted to show.

          \item For the second portion of the problem, if $f$ is twice continuously differentiable then $D^2f(x^*)$ is symmetric and can be decomposed into $C \Lambda C^{T}$ for $\Lambda$ a diagonal matrix of eigenvalues and $C$ an orthonormal matrix of eigenvectors (i.e. $C C^T = C^T C = I$). The entries of $\Lambda$ are all negative iff $D^2f(x^*)$ is negative definite. Hence for any non-zero $v$ we have
            \begin{align*}
              v^T D_x f(x^*) v
              =
              v^T C \Lambda C^T v
              =
              u^T \Lambda u
              =
              \sum_{i} \lambda_i u_i^2
              \le
              \lambda \sum_{i} u_i^2
              =
              \lambda v^T C C^T v
              =
              \lambda v^T v
            \end{align*}

            for $0 > \lambda \equiv \max_i \lambda_i$ and $u = C^T v$. Note we have $\le$ because $\lambda_i < 0$. Now
            \begin{align*}
              f(z) - f(x^*)
              &
              =
              \dfrac{1}{2}
              \dfrac{v^T}{\Fnorm{v}} D_x f(x^*) \dfrac{v}{\Fnorm{v}} \varepsilon^2
              +
              h(\varepsilon) \varepsilon^2
              \\
              &
              \le
              \dfrac{\lambda}{2}
              \dfrac{v^T v}{\Fnorm{v}^2} \varepsilon^2
              +
              h(\varepsilon) \varepsilon^2
              \\
              &
              =
              \left(
                \dfrac{\lambda}{2} + h(\varepsilon)
              \right) \varepsilon^2
            \end{align*}

            since $v^T v = \Fnorm{v}^2$ by definition. Finally, $h(\varepsilon) \to 0 \implies \exists \varepsilon > 0$ small enough s.t. $h(\varepsilon) < -\lambda/2$ (again, $\lambda < 0$), or simply $h(\varepsilon) + \lambda/2 < 0$. Since $\varepsilon^2 > 0$ we have
            \begin{align*}
              f(z) - f(x^*)
              &
              <
              0
            \end{align*}

            for all $z \in B_{\varepsilon}(x^*)$, showing $x^{*}$ is a strict local maximum.
        \end{enumerate}

      \item \textit{If $f$ is concave then $f(x + z) \le f(x) + z^T Df(x)$ for any $x, z$.}

        \solution Take $x, z$ s.t. $x$ and $x + z$ are in the domain of $f$. Let $g(\alpha) \equiv f(x + \alpha z)$ and, by the chain rule,
          \begin{align*}
            g^\prime(0)
            =
            z^T D_x f(x)
          \end{align*}

          (Note $x, x + z$ in the domain and $\alpha \in [0, 1]$ imply that $x + z\alpha$ in the domain since it's a convex set.) Now take a second order Taylor expansion:
          \begin{align*}
            g(\alpha)
            &
            =
            g(0)
            +
            g^{\prime}(0) \alpha
            +
            \dfrac{1}{2}
            g^{\prime\prime}(c) \alpha^2
          \end{align*}

          for some $c \in (0, \alpha)$. But
          \begin{align*}
            g^{\prime\prime}(c) \alpha^2
            =
            z^T D^2 f(x + cz) z \alpha^2 \le 0
          \end{align*}

          since $f$ concave $\implies D^2f$ is negative semi-definite (and $\alpha^2 \ge 0$). Finally,
          \begin{align*}
            f(x + \alpha z)
            &
            \le
            f(x)
            +
            z^T D_x f(x) \alpha
          \end{align*}

          Set $\alpha = 1$ and  we get the result. 

          \begin{remark}
          If, like me, you tried using the definitions, here's another solution, using limits:
          \begin{align*}
            g^\prime(0)
            =
            \lim_{\alpha \to 0}
            \dfrac{g(\alpha) - g(0)}{\alpha}
            =
            \lim_{\alpha \to 0}
            \dfrac{f(x + \alpha z) - f(x)}{\alpha}
          \end{align*}


          Therefore, if we can show that
          \begin{align*}
            f(x + z)
            &
            \le
            \dfrac{f(x + \alpha z) - f(x)}{\alpha} + f(x)
          \end{align*}

          then we have our result (the inequality is preserved in the limit). Note the above holds $\forall \alpha \ge 0$ if
          \begin{align*}
            \alpha f(x + z)
            &
            \le
            f(x + \alpha z) - f(x) + \alpha f(x)
            \\
            f(x + \alpha z \pm \alpha x)
            &
            \ge
            \alpha f(x + z) + (1 - \alpha) f(x)
            \\
            f(\alpha (x + z) + (1 - \alpha) x)
            &
            \ge
            \alpha f(x + z) + (1 - \alpha) f(x)
          \end{align*}

          This the last inequality is true by the definition of concavity of $x$. Note there is one subtlety here in that we haven't said anything about $\alpha < 0$. In this case, if $f$ is differentiable at $x$ then $g$ is differentiable at $0$, so the limit as $\alpha \to 0^+$ equals the limit as $\alpha \to 0^-$ (in this case you couldn't say anything about concavity since $\alpha \ge 0$ is required, but it's sufficient to take the right-limit case). Finally, we have
          \begin{align*}
            f(x + z)
            &
            \le
            \dfrac{f(x + \alpha z) - f(x)}{\alpha} + f(x)
            \\
            f(x + z)
            &
            \le
            \lim_{\alpha \to 0^+}
            \dfrac{f(x + \alpha z) - f(x)}{\alpha} + f(x)
            \\
            &
            =
            \lim_{\alpha \to 0}
            \dfrac{f(x + \alpha z) - f(x)}{\alpha} + f(x)
            \\
            &
            =
            z^T D_x f(x) + f(x)
          \end{align*}
          \end{remark}

      \item \textit{If $f$ is concave then any critical point (i.e. $x$ s.t. $Df(x) = 0$) is a global maximizer.}

        \solution Following the previous result, for any $x, z$ s.t. $x, x + z$ in the domain of $f$ we have
        \[
          f(x + z) \le f(x) + z^T D f(x)
        \]

        Let $x^*$ be a critical point and $y$ be any other point in the domain. Take $z \equiv y - x$ so that $x + z = y$ in the domain of $f$. Since $D f(x^*) = 0$, we find
        \[
          f(y)
          =
          f(x^* + (y - x^*))
          \le f(x^*) + (y - x^*)^T D f(x^*)
          = f(x^*) + (y - x^*)^T 0
          = f(x^*)
        \]

        Since $f(x^*) \ge f(y)$ for every $y$ in the domain, $x^*$ is a global maximizer.
    \end{enumerate}

    \begin{remark}
      We focused on maxima but the proofs for minima are analogous.
    \end{remark}

  \item {\itshape
    Define the set $\Delta = \set{p \in \mathbb{R}^L_{+}: \sum^{}_{l} p_l = 1}$ and the function $z^+$ on $\Delta$ as $z^+_l(p) = \max\set{z_l(p), 0}$, where $z(p) = \set{z_1(p), z_2(p), \ldots, z_L(p)}$ is a continuous function, homogeneous of degree 0, and satisfying $p \cdot z(p) = 0$ for all $p \in \mathbb{R}^L$. Denote $\alpha(p) = \sum^{}_{l} \left[p_l + z_l^+\right]$.}

    \begin{enumerate}[a)]
      \item \textit{Show that $\Delta$ is a non-empty compact and convex set.}

        \solution For non-emptiness, note
        \[
          p_l = \dfrac{1}{L}
          \implies
          \sum_l p_l = \sum^{}_{l} \dfrac{1}{L} = 1
        \]

        We next show that it is convex. Take any $p, q \in \Delta$ and let $r = \alpha p + (1 - \alpha) q$ for any $\alpha \in [0, 1]$. We can see that
        \[
          \sum^{}_{l} r_l
          = \sum^{}_{l} \left[\alpha p_l + (1 - \alpha) q_l\right]
          = \alpha \sum^{}_{l} p_l + (1 - \alpha) \sum^{}_{l} q_l
          = \alpha + (1 - \alpha)
          = 1
        \]

        Hence $r \in \Delta$, and $\Delta$ is convex. Then the set is bounded below by $0$, since $p_l \ge 0$ for each $l$, and above by $1$, since $p_j \le p_j + \sum^{}_{l \ne j} p_l \le 1$ (note the sum is positive since $p_l \ge 0$ for each $l$). If we can show that $\Delta$ is closed, then we can show that it is compact. (The cleanest proof I saw was to note $g: \mathbb{R}^L \to \mathbb{R}$ defined by $g(p) = \sum^{L}_{l = 1} p_l$ is continuous; then $g^{-1}(\set{1}) = \Delta$ is closed since $\set{1}$ is closed.) For my original proof, take any $q \notin \Delta$. We know that
        \[
          \sum^{}_{l} q \ne 1
          \implies
          \sum^{}_{l} q < 1
          \quad
          \text{ or }
          \quad
          \sum^{}_{l} q > 1
        \]

        We want to show that for any such $q$ there is some $\varepsilon$ s.t. every element $r \in B_{\varepsilon}(q)$ is not in $\Delta$. Suppose $\sum^{}_{l} q < 1$.  It must be that $0 < \delta < 1 - \sum^{}_{l} q_l$ for some $\delta$. Take $0 < \varepsilon < \delta / L$ and $r \in B_{\varepsilon}(q)$. We have that $0 \le r_l \le q_l + \varepsilon < q_l + \delta / L$ for each $l$, and so
        \[
          \sum^{}_{l} r_l \le \sum^{}_{l}  \left[q_l + \delta / L\right] = \sum^{}_{l}  q_l + \delta < 1
        \]

        Suppose $\sum^{}_{l} q > 1$. Then $0 < \delta < \sum^{}_{l} q - 1$ for some $\delta$. Take $0 < \varepsilon < \delta / L$ and $r \in B_{\varepsilon}(q)$. We have that $q_l - \delta / L \le q_l - \varepsilon \le r_l$ for each $l$, and so
        \[
          \sum^{}_{l} r_l \ge \sum^{}_{l}  \left[q_l - \delta / L\right] = \sum^{}_{l} q_l - \delta > 1
        \]

        Hence if $\sum^{}_{l} q \ne 1$, there is some $\varepsilon > 0$ s.t. for any $r \in B_{\varepsilon}(q)$, $r \notin \Delta$, which means that $\Delta^C$ is open, and thus $\Delta$ is closed. Since $\Delta$ is closed and bounded, it is compact.

      \item \textit{Show that $f: \Delta \to \Delta$ is continuous in $p$.}
        \[
          f(p) = \dfrac{1}{\alpha(p)} \left(p + z^+(p)\right)
        \]

        \solution We will use the result that if $g: A \to B$ is continuous and $h: B \to C$ is continuous then the composition $h \circ g$ is continuous. Clearly $f_1(x) = x$ the identity function is continuous. Since $f_2(x) = \max\set{x, 0}$ is continuous and $z_l(p)$ is continuous, $z_l^+(p) = f_2(z(p))$ is continuous. Hence $p_l + z_l^+(p)$ is continuous for each $l$, which means $p \mapsto p + z^+(p)$ will be continuous.

        Since $p_l + z_l^+(p)$ is continuous, $\alpha(p) = \sum^{}_{l} p_l + z_l^+(p)$, the linear combination of $L$ continuous functions, is continuous. To finish, we have that $f_3(x) = 1 / x$ is continuous when $x \ne 0$, so $f_3(\alpha(p))$ will be continuous whenever $\alpha(p) \ne 0$. However, we have that
        \[
          \alpha(p) = \sum^{}_{l} p_l + \sum^{}_{l} z^+_l(p)
          \quad\quad
          \sum^{}_{l} p_l = 1
          \quad\quad
          z^+_l(p) = \max\set{z_l(p), 0} \ge 0
        \]

        Hence $\alpha(p) \ge 1 > 0$. This means that $f(p)$ is the composition of continuous functions, and thus itself is continuous.

      \item \textit{Prove that $f$ has a fixed point. (Hint: You can use existing theorems!)}

        \solution This follows directly from Brouwer's fixed point theorem. Any continuous $f: \Delta \to \Delta$ with $\Delta \subseteq \mathbb{R}^N$ non-empty, convex, and compact has a fixed point. I think we just need to show that the co-domain of $f$ is, indeed, $\Delta$. Here note that for any $q = f(p)$, we have
        \[
          \sum^{}_{l} q_l = \dfrac{1}{\sum^{}_{l} \left[p_l + z^+_l\right]} \sum^{}_{l} \left[p_l + z^+_l\right] = 1
        \]

      \item \textit{Use the fact $f$ has a fixed point and the properties of $z$ to argue that $\exists p^*$ s.t. $z^+(p^*) \cdot z(p^*) = 0$. (Hint: Use the fact $p^* \cdot z(p^*) = 0$.)}

        \solution For $p^* \in \Delta$ a fixed point of $f$ we have
        \begin{equation}
          \begin{array}{rl}
            p^*
            & = f(p^*) = \dfrac{1}{\alpha(p^*)} \left(p^* + z^+(p^*)\right)
            \\
            p^* \cdot z(p^*)
            & = \dfrac{1}{1 + \sum^{}_{l} z^+(p^*)} \left(p^* + z^+(p^*)\right) \cdot z(p^*)
            \\
            0
            & = \dfrac{1}{1 + \sum^{}_{l} z^+(p^*)} \left(p^* \cdot z(p^*) + z^+(p^*) \cdot z(p^*)\right)
            \\
            0
            & = z^+(p^*) \cdot z(p^*)
          \end{array}
          \nonumber
        \end{equation}

      \item \textit{Conclude thet $z(p^*) \le 0$.}

        \solution
        \[
          z^+(p^*) \cdot z(p^*)
          =
          \sum^{}_{l} z_l(p^*) \max\set{z_l(p^*), 0}
        \]

        If $z_l(p^*) \le 0$ for any $l$, then $z_l(p^*) \max\set{z_l(p^*), 0} = 0$. Hence
        \[
          z^+(p^*) \cdot z(p^*)
          =
          \sum^{}_{l: z_l > 0} z_l(p^*) \max\set{z_l(p^*), 0}
        \]

        But if $z_l > 0$ for any $l$l, then $z_l(p^*) \max\set{z_l(p^*), 0} = z_l(p^*)^2 > 0$, which means
        \[
          z^+(p^*) \cdot z(p^*) > 0
        \]

        a contradiction. Thus $z_l(p^*) \le 0$.
    \end{enumerate}

  \begin{remark}
    If for consumer $i$ we define the excess demand function $z_i(p) = x_i(p, \omega_i) - \omega_i$ for wealth $\omega_i$ and prices $p$. One way to define general equilibrium is vector of prices s.t. $\sum^{}_{i} z_i(p) \le 0$ for all $i$ (i.e. there is no aggregate excess demand). You have just shown that under some conditions such a price vector always exists.
  \end{remark}

  \item \textit{Use the chain rule and the FTC to prove the Leibniz rule:}
    \[
      \dfrac{d}{dx} \int_{u(x)}^{v(x)} f(t) dt
      =
      f(v(x)) \dfrac{dv}{dx}
      - f(u(x)) \dfrac{du}{dx}
    \]

    \solution It hadn't quite dawned on me I have this easier version of Leibniz' rule. I have a note in the end about the full form. Now from the FTC since $f$ integrable and continuous on $[a, b]$ we have for any $x \in [a, b]$
    \[
      F(x) = \int_{a}^{x} f(t) dt
    \]

    is continuously differentiable on $(a, b)$ and
    \[
      \dfrac{d}{dx} F(x) = f(x)
    \]

    Further,
    \[
      F(b) - F(a) = \int_{a}^{b} f(t) dt
    \]

    If $f$ is integrable on $[u(x), v(x)]$, then we have that
    \[
      \int_{u(x)}^{v(x)} f(t) dt
      =
      F(v(x)) - F(u(x))
    \]

    Now we combine the FTC and the chain rule to see that
    \[
      \dfrac{d}{dx} \int_{u(x)}^{v(x)} f(t) dt
      =
      \dfrac{d}{dx} F(v(x)) - \dfrac{d}{dx} F(u(x))
      =
      f(v(x)) \dfrac{dv}{dx} - f(u(x)) \dfrac{du}{dx}
    \]

    For the full version we have something similar, except $f$, and therefore $F$, can depend on $x$ as well. In this case, we follow similar steps. However, it will be useful to make a distinction when taking partial derivatives that we evaluate the result at $u(x), v(x)$ after taking partials. Calling the first argument $y$ for the sake of this:
    \begin{align*}
      \dfrac{d}{dx} \int_{u(x)}^{v(x)} f(x, t) dt
      &
      =
      \dfrac{d}{dx}
      F(x, v(x))
      -
      \dfrac{d}{dx}
      F(x, u(x))
      \\
      &
      =
      \left[
        \dfrac{\partial}{\partial t} F(y, t)
      \right]_{(y, t) = (x, v(x))}
      \dfrac{dv(x)}{dx}
      -
      \left[
        \dfrac{\partial}{\partial t} F(y, t)
      \right]_{(y, t) = (x, u(x))}
      \dfrac{du(x)}{dx}
      \\
      &
      \quad
      +
      \left[
        \dfrac{\partial}{\partial y}
        F(y, t)
      \right]_{(y, t) = (x, v(x))}
      -
      \left[
        \dfrac{\partial}{\partial y}
        F(y, t)
      \right]_{(y, t) = (x, u(x))}
    \end{align*}

    The first two terms simplify in the same way from the FTC, and for the second term,
    \begin{align*}
      \left[
        \dfrac{\partial}{\partial y}
        F(y, t)
        -
        \dfrac{\partial}{\partial y}
        F(y, t)
      \right]_{(y, t) = (x, u(x))}^{(y, t) = (x, v(x))}
      =
      \int_{u(x)}^{v(x)} 
      \dfrac{\partial}{\partial x} f(x, t) dt
    \end{align*}

    because we take derivatives with respect to the first argument holding the second argument constant, so the partial derivative can ``go through'' the integral as neither the limits nor the function we are integrating with respect to are changing.
\end{enumerate}

%----------------------------------------------------------------------
\end{document}
