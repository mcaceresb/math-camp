%----------------------------------------------------------------------
\documentclass{article}

\usepackage[homework]{brownpreamble}
\lhead{\color{light-gray} \itshape Math Camp 2021}
\rhead{\color{light-gray} \itshape Suggested Solutions 2}
\renewcommand\sectiontype{Suggested Solutions \thesection\ }
% \renewcommand\thesection{\arabic{section}}
% \setcounter{section}{0}

%----------------------------------------------------------------------
\begin{document}
\displayoptions

% ---------------------------------------------------------------------
\section{}

\begin{enumerate}[1.]
  \item \textit{Using the fact that every Cauchy sequence converges, prove the Bolzano-Weierstrass theorem. (Hint: First show every bounded sequence admits a monotonic sub-sequence. Second, show bounded monotonic sequences are Cauchy.)}

    \solution The Bolzano-Weierstrass theorem states every bounded sequence has a convergent sub-sequence. Our strategy to prove it using Cauchy sequences will be:
    \begin{itemize}[label=$\bullet$]
      \item Show that every bounded sequence admits a monotonic sub-sequence.

      \item Show every bounded monotonic sequence is Cauchy.
    \end{itemize}

    For the first part, the claim is, formally,
    \begin{itemize}[label=$\bullet$]
      \item For each $m$ there exists some $n > m$ s.t. $x_m \ge x_n$

      \item \textit{Or} for each $k$ there exists some $n > k$ s.t. $x_k \le x_n$.
    \end{itemize}

    Maybe both are true, but if at least one is true then we have the first step. By contradiction, suppose this claim is false, meaning both statements must be false. That is,
    \begin{itemize}[label=$\bullet$]
      \item $\exists m_1$ s.t. $\forall n > m_1$ we have $x_{m_1} < x_n$

      \item \textit{and} $\exists k_1$ s.t. $\forall n > k_1$ we have $x_{k_1} > x_n$.
    \end{itemize}

    In other words, for $n \ge M_1 \equiv \max\set{m_1, k_1}$ we have $x_n \in (x_{m_1}, x_{k_1})$. Now we can iterate the claim: Suppose there is no monotonic sub-sequence of $x_m$ for $m \ge M_1$. Then
    \begin{itemize}[label=$\bullet$]
      \item $\exists m_2$ s.t. $\forall n > m_2$ we have $x_{m_2} < x_n$

      \item and $\exists k_2$ s.t. $\forall n > k_2$ we have $x_{k_2} > x_n$.
    \end{itemize}

    Note $m_2, k_2$ must each be larger than $M_1$, so $x_{m_2}, x_{k_2} \in (x_{m_1}, x_{k_1})$, and for $n \ge M_2 \equiv \max\set{m_2, k_2}$ we have $x_n \in (x_{m_2}, x_{k_2}) \subset (x_{m_1}, x_{k_1})$. We can proceed by induction, and construct a sequence (two sequences, actually) $x_{m_l}$ strictly increasing and $x_{k_l}$ strictly decreasing. These are strictly monotonic sub-sequences of $x_m$, which is a contradiction.

    For the second step, we again by contradiction suppose that monotonic sequences are not Cauchy. Remember the definition of Cauchy is for every $\varepsilon > 0$ there exists some $M$ s.t. $d(x_k, x_n) < \varepsilon$ whenever $n, k \ge M$. If a sequence is not Cauchy, then $\exists \varepsilon > 0$ s.t. for each $k, n$ we have
    \begin{align*}
      d(x_k, x_n) \ge \varepsilon
    \end{align*}

    Consider $x_{m_k}$ any monotonic sub-sequence of $x_m$ and suppose it is increasing. We know that sequential elements of this sub-sequence are at least $\varepsilon$ away from each other, since this is true of any two elements of $x_m$. Hence we can write
    \begin{align*}
      x_{m_1} + \varepsilon
      \le
      x_{m_2}
      \quad
      x_{m_2} + \varepsilon
      \le
      x_{m_3}
      \quad
      x_{m_1} + 2\varepsilon
      \le
      x_{m_3}
      \quad
      \quad
      \cdots
    \end{align*}

    In general, we can find that
    \begin{align*}
      x_{m_1} + (K - 1) \varepsilon \le x_{m_K}
    \end{align*}

    for any $K$. Since $\varepsilon > 0$, take any upper bound of the sequence $U$. We have that
    \begin{align*}
      K > \dfrac{U - x_{m_1}}{\varepsilon} + 1
      \implies
      x_{m_1} + (K - 1) \varepsilon > U
      \implies
      x_{m_K} > U
    \end{align*}

    meaning $U$ is not an upper bound, contradiction. The proof for a decreasing sub-sequence is analogous. Thus we have shown every bounded sequence admits a monotonic sub-sequence; this sub-sequence is Cauchy, and since Cauchy sequences converge this sub-sequence converges.

  \item {\itshape
    Let $(x_m)$ be any sequence. We define
    \[
      \limsup_{m \to \infty} = \lim_{m \to \infty} \left(\sup_{k \ge m} x_k\right)
    \]

    and
    \[
      \liminf_{m \to \infty} = \lim_{m \to \infty} \left(\inf_{k \ge m} x_k\right)
    \]

    Consider $(x_m) \in \mathbb{R}$ bounded. Show $x_m \to x \iff \limsup x_m = \liminf x_m = x$.}

    \solution First we do $\implies$, that is, $x_m \to x$. We want to show that $\limsup x_m = \liminf x_m = x$. By definition,
    \[
      \limsup_{m \to \infty} x_m
      =
      \lim_{m \to \infty} \sup_{k \ge m} x_k
    \]

    We want to show that for every $\varepsilon > 0$, there is some $M$ s.t. $m > M$ gives
    \[
      |\sup_{k \ge m} x_k - x| < \varepsilon
      \quad\quad
      x - \varepsilon < \sup_{k \ge m} x_k < x + \varepsilon
    \]

    Suppose not; since $x_m \to x$, where is some $M$ s.t. $|x_m - x| < \varepsilon$. Hence for $m > N, m > M$, we have
    \[
      \sup_{k \ge m} x_k < x - \varepsilon < x_m
      \text{ or }
      x_m < x + \varepsilon < \sup_{k \ge m} x_k
    \]

    The first statement is not true, or $\sup_{k \ge m} x_k$ cannot be the $\sup$, and the second statement cannot be true, or $x + \varepsilon$ would instead be the $\sup$. Hence $|\sup_{k \ge m} x_k - x| < \varepsilon$ for $m > \max\set{M, N}$. The argument for $\inf$ is completely analogous; if we assume the $\inf$ does not converge to $x$, then we find $\widetilde{\varepsilon} > 0, \widetilde{M}, \widetilde{N}$ s.t.
    \[
      \inf_{k \ge m} x_k < x - \widetilde{\varepsilon} < x_m
      \text{ or }
      x_m < x + \widetilde{\varepsilon} < \inf_{k \ge m} x_k
    \]

    for $m > \max\set{\widetilde{M}, \widetilde{N}}$. Both statements are, again, false: The first because it means $x - \widetilde{\varepsilon}$ would tbe the $\inf$, and the second because the $\inf$ cannot be greater than $x_m$.

    Now we show $\impliedby$.  We want to show that for every $\varepsilon > 0$, there is some $M$ s.t.
    \[
      m > M \implies |x_m - x| < \varepsilon
    \]

    We have that
    \[
      \sup_{k \ge m} x_k \to x
      \text{ and }
      \inf_{k \ge m} x_k \to x
    \]

    So for $\varepsilon > 0$, there is some $M_1, M_2$ s.t.
    \[
      m > M_1 \implies |\sup_{k \ge m} x_k - x| < \varepsilon
      \text{ and }
      m > M_2 \implies |\inf_{k \ge m} x_k - x| < \varepsilon
    \]

    Take $M > \max\set{M_1, M_2}$. Then we have
    \[
      x - \varepsilon < \inf_{k \ge m} x_k < x + \varepsilon
      \text{ and }
      x - \varepsilon < \sup_{k \ge m} x_k < x + \varepsilon
    \]

    To finish, we simply need to note that
    \[
      \inf_{k \ge m} x_k \le x_m \le \sup_{k \ge m} x_k
    \]

    by definition of $\inf$ and $\sup$. Hence we have
    \[
      x - \varepsilon < \inf_{k \ge m} x_k \le x_m \le \sup_{k \ge m} x_k < x + \varepsilon
      \implies
      x - \varepsilon < x_m < x + \varepsilon
    \]

    which is what we wanted.

  \item {\itshape
    Given a set $A \subseteq \mathbb{R}$ we say that a function $f: S \to \mathbb{R}$ is \keyword{uniformly continuous} if $\forall \varepsilon > 0 ~~ \exists \delta > 0$ s.t. $\forall x, y \in S$
    \[
      |x - y| < \varepsilon \implies |f(x) - f(y)| < \varepsilon
    \]

    The main difference between this and the usual definition of continuity, is that the latter can have different values of $\delta$ given $x, y \in S$. If a function is uniformly continuous, we need $\delta$ picked for any $x, y$ (although it might depend on $\varepsilon$.)}
    \begin{enumerate}[a)]
      \item \textit{Take $f(x) = 1 / x$ for $f: (0, 1) \to \mathbb{R}$. Show $f$ is continuous.}

        \solution First, note for any $x \in (0, 1)$ and $\delta > 0$ small enough we have
        \begin{align*}
            |x - a| < \delta
            &
            \implies
            \dfrac{1}{a}
            \in
            \left(
              \dfrac{1}{x + \delta},
              \dfrac{1}{x - \delta}
            \right)
            \\
            &
            \implies
            \left|\dfrac{1}{x} - \dfrac{1}{a}\right|
            =
            \left|\dfrac{x - a}{xa}\right|
            <
            \left|\dfrac{1}{x} - \dfrac{1}{a}\right|
            <
            \left|\dfrac{\delta}{x (x - \delta)}\right|
        \end{align*}

        Let $\delta \equiv \varepsilon x^2 / (1 + \varepsilon x)$ (noting $x \in (0, 1)$). We have that
        \[
            |x - a| < \delta
            \implies
            \left|\dfrac{1}{x} - \dfrac{1}{a}\right|
            <
            \left|\dfrac{\delta}{x (x - \delta)}\right|
            =
            \dfrac{\dfrac{\varepsilon x^2}{1 + \varepsilon x}}{x \left(x - \dfrac{\varepsilon x^2}{1 + \varepsilon x}\right)}
            =
            \varepsilon
        \]

        which shows that $1/x$ is continuous.

      \item \textit{Show that if for two sequences $(x_m), (y_m) \subseteq S$ with $\lim_{m \to \infty} (x_m - y_m) = 0$ $\exists \varepsilon > 0$ s.t. $|g(x_m) - g(y_m)| > \varepsilon ~~ \forall m \in \mathbb{N}$, then $g: S \to \mathbb{R}$ is not uniformly continuous.}

        \solution First, note every $\varepsilon_1$ there is some $M$ s.t.
        \[
            m > M \implies |x_m - y_m| < \varepsilon_1
        \]

        $g: A \to \mathbb{R}$ is uniformly continuous if for every $\varepsilon_2$ there is some $\delta$ such that for all $x, y \in A$,
        \[
            |x - y| < \delta \implies |g(x) - g(y)| < \varepsilon_2
        \]

        Suppose $g(\cdot)$ is uniformly continuous and let $\varepsilon_1 = \delta, \varepsilon_2 = \varepsilon$. Then we know there is some $M$ s.t. $m > M$ gives $|x_m - y_m| < \delta$, which in turn would imply $|g(x_m) - g(u_m)| < \varepsilon$. However, we assumed $\varepsilon < |g(x_m) - g(u_m)|$, which is a contradiction.

      \item \textit{Use the result above to check that $f$ defined in (a) is not uniformly continuous.}

        \solution Let $x_m = 1 / m$ and $y_m = 1 / (m + 1)$. Clearly,
        \[
          \dfrac{1}{m} - \dfrac{1}{m + 1}
          =
          \dfrac{m + 1 - m}{m (m + 1)}
          =
          \dfrac{1}{m (m + 1)}
          \to
          0
        \]

        However,
        \[
          \left|f\left(\dfrac{1}{m}\right) - f\left(\dfrac{1}{m + 1}\right)\right|
          =
          \left|\dfrac{1}{1 / m} - \dfrac{1}{1 / (m + 1)}\right|
          =
          \left|m - (m + 1)\right|
          =
          1
          >
          \varepsilon
        \]

        for, say, $\varepsilon = 1 / 2 > 0$. By (b) we know then that $f$ is not uniformly continuous.

      \item \textit{Show that if $(x_m) \in S$ is Cauchy, then $(y_m)$ defined by $y_m = h(x_m)$ is also Cauchy when $h: S \to \mathbb{R}$ is uniformly continuous.}

        \solution If $h(\cdot)$ uniformly continuous, we know for every $\varepsilon > 0$ there is some $\delta > 0$ s.t.
        \[
          |x - y| < \delta \implies |h(x) - h(y)| < \varepsilon
        \]

        If $x_m$ is Cauchy, we know that for this $\delta > 0$, we can find some $M$ s.t.
        \[
          m > M \implies |x_m - x_{m + 1}| < \delta
        \]

        Hence for $m > M$,
        \[
          |x_m - x_{m + 1}| < \delta
        \]

        Since $h()$ is uniformly continuous, this in turn implies
        \[
          |h(x_m) - h(x_{m + 1})| < \varepsilon
        \]

        Which means that for any $\varepsilon$ we can find some $M$ s.t.
        \[
          |y_m - y_{m + 1}| < \varepsilon
        \]

      \item \textit{Check $x_m = 1 / m$ is Cauchy but $x_m = m$ is not.}

        \solution $x_m = m$ and $x_{m + 1} = m + 1$ gives $|x_m - x_{m + 1}| = |m - (m + 1)| = |1| = 1 > \varepsilon$ for all $0 < \varepsilon < 1$, so it cannot be Cauchy. Now
        \[
          \left|\dfrac{1}{m} - \dfrac{1}{m + 1}\right|
          =
          \left|\dfrac{m + 1 - m}{m (m + 1)}\right|
          =
          \left|\dfrac{1}{m (m + 1)}\right|
        \]

        For any $\varepsilon > 0$, let
        \[
          M = \Fceil{\dfrac{1}{\varepsilon}} + 1 > \dfrac{1}{\varepsilon}
          \implies
          \varepsilon > \dfrac{1}{M}
          \text{ and }
          \varepsilon > \dfrac{1}{M + 1}
        \]

        Hence for any $m > M$,
        \[
          \left|\dfrac{1}{m (m + 1)}\right|
          < \varepsilon^2
          \le \varepsilon
        \]

        given $\varepsilon \le 1$ (and if $\varepsilon > 1$ then the inequality holds for all $m$).

      \item \textit{Use the sequences in (e) and the result in (d) to give an alternative proof that the function defined in (a) is not uniformly continuous.}

        \solution If $f(\cdot)$ is uniformly continuous, then $y_m \equiv f(x_m)$ is Cauchy for any $x_m$ that is Cauchy. However, we just proved that $x_m = 1 / m$ is Cauchy while $y_m = 1 / (1 / m) = m$ is not Cauchy, which is a contradiction. Hence $f(\cdot)$ cannot be uniformly continuous.
    \end{enumerate}

  \item {\itshape
    A function $f: \mathbb{R}^N \to \mathbb{R}$ is homogeneous of degree $r \in \mathbb{Z}$ if $\forall t > 0, x \in \mathbb{R}^N$ we have
    \[
      f(tx) = t^r f(x)
    \]
    }

    \begin{enumerate}[a)]
      \item \textit{Show if $f(x)$ is homogeneous of degree $r$ the partial derivative is homogeneous of degree $r - 1$.}

        \solution If we can assume that the partial derivative exists, then we have

        \begin{align*}
          \dfrac{\partial}{\partial x_j} f(tx) & = t^r \dfrac{\partial}{\partial x_j} f(x)
          \\
          t \dfrac{\partial}{\partial x_j} f(tx) & = t^r \dfrac{\partial}{\partial x_j} f(x)
          \\
          \dfrac{\partial}{\partial x_j} f(tx) & = t^{r - 1} \dfrac{\partial}{\partial x_j} f(x)
        \end{align*}

        which is the definition of homogeneity of degree $r - 1$.

      \item {\itshape
          Show that if $f(x)$ is homogeneous of degree $r$ and differentiable  then for any $\widetilde{x}$ we have
        \[
          \sum^{N}_{n = 1} \dfrac{\partial f(\widetilde{x})}{\partial x_n} \widetilde{x}_n = r f(\widetilde{x})
        \]

        that is, $\nabla f(\widetilde{x}) \cdot \widetilde{x} = r f(\widetilde{x})$.}

      \solution Take derivatives with respect to $t$, then we have
        \[
          \dfrac{d}{dt} f(tx)
            = \dfrac{d}{dt} t^r f(x)
          \quad\quad
          \sum^{N}_{n = 1}
          \left(\dfrac{\partial}{\partial x_n} f(tx)\right)
          \underbrace{\left(\dfrac{d}{dt} (t x_n)\right)}_{x_n}
            = r t^{r - 1} f(x)
        \]

        is true for every $t$. So simply choose $t = 1$, and we get
        \[
          \sum^{N}_{n = 1} \dfrac{\partial}{\partial x_n} f(x) x_n = r f(x)
        \]

        which is what we wanted.\footnote{I think the only sticky point here is to show that
        \[
          \dfrac{d}{dt}f(tx) = \nabla f(tx) \cdot \dfrac{d}{dt} tx
        \]

        which is true by the chain rule.}
    \end{enumerate}

  \item {\itshape
    Take the simplified IS-LM system of equations
    \[
      Y - C(Y - T) - I(r) = G
      \quad\quad
      M^D(Y, r) = M^S
    \]

    Suppose that $0 < C^\prime(x) < 1, I^\prime(r) < 0, \dfrac{\partial M}{\partial Y} > 0$ and $\dfrac{\partial M}{\partial r} < 0$.}
    \begin{enumerate}[a)]
      \item \textit{Use the IFT to check that one can represent the endogenous variables $Y, r$ as a function of the exogenous variables $G, M^S, T$.}

        \solution Note that
          \begin{align*}
            Y - C(Y - T) - I(r) - G & = 0 \\
            M^S - M^D (Y, r)        & = 0
          \end{align*}

          with endogenous variables $x \equiv (Y, r)$, national income and the interest rate, and exogenous variables $\theta \equiv (M^S, G, T)$, money supply, government spending, and taxes.
          \begin{equation}
           \label{eq:is_lm_example}
            f(\theta, x)
            =
            \left[\begin{matrix}
              f_1(\theta, x) \\
              f_2(\theta, x)
            \end{matrix}\right]
            =
            \left[\begin{matrix}
              Y - C(Y - T) - I(r) - G \\
              M^S - M^D (Y, r)
            \end{matrix}\right]
            = 0
          \end{equation}

          is the exact type of equation where IFT applies. If $D_x$ is non-singular, then for some open neighborhood of $\theta$ and some $h$ we can write
          \begin{equation}
            \begin{array}{c}
              h(\theta)
              = \left[\begin{matrix}
                Y(M^S, G, T) \\
                r(M^S, G, T)
              \end{matrix}\right] \\[6pt]
              D_\theta f(\theta, h(\theta)) + D_x f(\theta, h(\theta)) D_\theta h(\theta) = 0
            \end{array}
            \label{eq:ift_example_statement}
          \end{equation}

          We find that
          \begin{align*}
            D_\theta f(\theta, h(\theta))
            & =
            \left[\begin{matrix}
                0
              & -1
              & C^\prime(Y(\cdot) - T) \\[6pt]
                1
              & 0
              & 0
            \end{matrix}\right] \\[6pt]
            D_x f(\theta, h(\theta))
            & =
            \left[\begin{matrix}
                \dfrac{\partial f_1}{\partial Y} = 1 - C^\prime(Y(\cdot) - T)
              & \dfrac{\partial f_1}{\partial r} = -I^\prime(r(\cdot)) \\[6pt]
                \dfrac{\partial f_2}{\partial Y} = -\dfrac{\partial M^D}{\partial Y}
              & \dfrac{\partial f_2}{\partial r} = -\dfrac{\partial M^D}{\partial r}
            \end{matrix}\right] \\[6pt]
            \left[D_x f(\theta, h(\theta))\right]^{-1}
            & =
            \dfrac{1}{D}
            \left[\begin{matrix}
                - \dfrac{\partial M^D}{\partial r}
              & I^\prime(r(\cdot)) \\[6pt]
                \dfrac{\partial M^D}{\partial Y}
              & 1 - C^\prime(Y(\cdot) - T)
            \end{matrix}\right] \\[6pt]
            D & =
            \underbrace{
             - \overbrace{\dfrac{\partial M^D}{\partial r}}^{< 0}
               \underbrace{\Bpar{1 - C^\prime(Y(\cdot) - T)}}_{> 0}
             }_{> 0}
            \underbrace{
             - \overbrace{I^\prime(r(\cdot))}^{< 0}
               \underbrace{\dfrac{\partial M^D}{\partial Y}}_{> 0}
             }_{> 0}
             \implies D > 0
          \end{align*}

        A non-zero determinant $\implies$ the inverse exists, and we can use \Cref{eq:ift_example_statement} to characterize $D_\theta h(\theta)$

      \item \textit{Check the sign of the effect of an infinitesimal increase in government spending $G$ on $r$ and $Y$ keeping $M^S$ and $T$ fixed.}

        \solution Using the results from the previous section, we have that
        \begin{align*}
          D_\theta h(\theta)
          & =
          \left[\begin{matrix}
              \dfrac{\partial Y}{\partial M^S}
            & \dfrac{\partial Y}{\partial G}
            & \dfrac{\partial Y}{\partial T} \\[6pt]
              \dfrac{\partial r}{\partial M^S}
            & \dfrac{\partial r}{\partial G}
            & \dfrac{\partial r}{\partial T}
          \end{matrix}\right]
          \\[6pt]
          & =
            -\dfrac{1}{D}
            \left[\begin{matrix}
                - \dfrac{\partial M^D}{\partial r}
              & I^\prime(r(\cdot)) \\[6pt]
                \dfrac{\partial M^D}{\partial Y}
              & 1 - C^\prime(Y(\cdot) - T)
            \end{matrix}\right]
            \left[\begin{matrix}
                0
              & -1
              & C^\prime(Y(\cdot) - T) \\[6pt]
                1
              & 0
              & 0
            \end{matrix}\right]
          \\[6pt]
          & =
          - \dfrac{1}{D}
          \left[\begin{matrix}
                I^\prime(r(\cdot))
            &   \dfrac{\partial M^D}{\partial r}
            & - \dfrac{\partial M^D}{\partial r} C^\prime(Y(\cdot) - T)
            \\[6pt]
                1 - C^\prime(Y(\cdot) - T)
            & - \dfrac{\partial M^D}{\partial Y}
            &   \dfrac{\partial M^D}{\partial Y} C^\prime(Y(\cdot) - T)
            \\[6pt]
          \end{matrix}\right] \\
          & =
          - \dfrac{1}{D}
          \left[\begin{matrix}
              < 0
            & < 0
            & > 0 \\[6pt]
              > 0
            & < 0
            & > 0
          \end{matrix}\right]
          =
          \dfrac{1}{D}
          \left[\begin{matrix}
              > 0
            & > 0
            & < 0 \\[6pt]
              < 0
            & > 0
            & < 0
          \end{matrix}\right]
        \end{align*}

      Which means that for some $x = (Y, r), \theta = (M^S, G, T)$ that satisfies \Cref{eq:is_lm_example} there is some local neighborhood around $\theta$ where we can characterize the behavior of $(Y, r)$ with respect to each of the variables in $\theta$. In particular, income reacts positively to increases money supply or government spending but negatively to taxes, while the interest rate goes down with increases in the money supply or taxes but goes up with increases in government spending.
    \end{enumerate}

  \item {\itshape
    Take the correspondence $f: \mathbb{R}^3_{++} \to \mathbb{R}^2_{++}$ (i.e. with strictly posisive arguments) defined by
    \[
      f(p_1, p_2, w)
      =
      \begin{cases}
        \left(\dfrac{w}{2p_1}, \dfrac{w}{2 p_2}\right)
          & \text{if } \dfrac{w^2}{4 p_1 p_2} \ne 1 \\[6pt]
        \varnothing
          & \text{if } \dfrac{w^2}{4 p_1 p_2} = 1
      \end{cases}
    \]}

    \begin{enumerate}
      \item \textit{Is $f$ upper hemicontinuous?}

        \solution No. Take
        \[
          p^m_1, p^m_2
          =
          \left(
            1 - \dfrac{1}{m},
            1 - \dfrac{1}{m}
          \right)
          \quad\quad
          w_m = 2
        \]

        Hence $(p^m_1, p^m_2, w^m) \to (1, 1, 2)$, and $f(1, 1, 2) = \varnothing$ because $2^2 / 4 = 1$. However
        \[
          y_m
          \equiv
          f(p^m_1, p^m_2, w^m)
          =
          \left(\dfrac{2}{2 (1 - 1 / m)}, \dfrac{2}{2 (1 - 1 / m)}\right)
          =
          \left(\dfrac{m}{m - 1}, \dfrac{m}{m - 1}\right)
          \to
          (1, 1)
        \]

        This means that $y_m \to 1$ and there is no sub-sequence of $y_m$ that converges to a point in the empty set.

      \item \textit{Is $f$ lower hemicontinuous?}

        \solution Yes. Take any $(p_1^m, p_2^m, w^m) \to (p_1, p_2, w)$. We want to show for each $y \in f(p_1, p_2, w)$ then $\exists y_m$ s.t. $y_m \to y$ with $y_m \in f(p_1^m, p_2^m, w^m)$. The trick here is to note
        \begin{align*}
          \dfrac{w}{2 p_1} < y < \dfrac{w}{2 p_2}
        \end{align*}

        means that  for some $\varepsilon > 0$ we have
        \begin{align*}
          \dfrac{w}{2 p_1} + \varepsilon < y < \dfrac{w}{2 p_2} - \varepsilon
        \end{align*}

        (in particular this works for any $\varepsilon < \min\set{y - (w / 2p_1), (w/2p_2) - y}$). By convergence, $\exists M_1$ s.t.
        \begin{align*}
          \dfrac{w^m}{2 p_1^m} < \dfrac{w}{2 p_1} + \varepsilon
          < 
          y 
          <
          \dfrac{w}{2 p_2} - \varepsilon
          <
          \dfrac{w^m}{2 p_2^m}
        \end{align*}

        and by the Archimedean principle (!) there is some $M_2$ s.t. $1 / M_2 < \varepsilon$. Let $M \equiv \max\set{M_1, M_2}$ and we have that the sequence $y_m = y + \dfrac{1}{m}$ is s.t.
        \begin{align*}
          \dfrac{w^m}{2 p_1^m} <
          < 
          y_m
          <
          \dfrac{w^m}{2 p_2^m}
        \end{align*}

        for any $m \ge M$. Since $1/m \to 0$ we have $y_m \to y$, which shows that $f$ is lhc. (Note there is no sticking point if $(p_1^m, p_2^m, w^m) \to (p_1, p_2, w)$ s.t. $w^2 / 4p_1p_2 = 1$ because then the correspondence is empty.)
    \end{enumerate}
\end{enumerate}

%----------------------------------------------------------------------
\end{document}
