%! TEX program = xelatex
%% Mauricio Caceres Bravo <mauricio.caceres.bravo@gmail.com>
%% 2022-04-20 17:20

%----------------------------------------------------------------------
\documentclass[11pt]{article}

\usepackage{nberpreamble}
\usepackage{changepage}
% \renewcommand\textsc[1]{{\fontfamily{put}\fontshape{sc}\selectfont#1}}

\title{\bfseries\scshape Math Camp}
\author{\scshape Brown University \\ \scshape Department of Economics}
\date{\scshape Summer 2022}

\renewcommand{\displayoptions}{\maketitle\pagenumbering{arabic}}
\titleformat{\section}{\scshape\bfseries\Large}{\thesection}{1em}{#1}
\setlength{\parindent}{1.5em}

%----------------------------------------------------------------------
\begin{document}
\displayoptions

{
  \bfseries
  \begin{adjustwidth}{\parindent}{0pt}
    Instructor: Mauricio C\'aceres Bravo \\
    E-mail: mauricio\_caceres\_bravo@brown.edu \\
    Office Hours: After class and by appointment \\[12pt]
    Times: 8/15, 8/16, 8/17, 8/18, 8/22, 8/23 from 9AM--12PM (10 min break) \\
    Location: Robinson Hall Room 301
    Website: \href{https://mcaceresb.github.io/tablefill}{mcaceresb.github.io/math-camp \ExternalLink.}
  \end{adjustwidth}
  \vspace{12pt}
}

% ---------------------------------------------------------------------
\section{Course Description}
\label{sec:course_description}

The goal of this course is to present many of the mathematical tools that you will typically encounter throughout your first year at the PhD program.

Keep in mind that this course should serve as a warm-up for the challenging first year that you’ll have. We expect that, at the end of the Math Camp, students will be familiarized with the theory presented and will be able to apply it throughout the first year.

Finally, many of the topics we will cover will be presented in a ``Cookbook'' way, perhaps without the details a rigorous and formal student ideally would want. A more rigorous presentation of the topics we will expose will come shortly (ECON 2010).

% ---------------------------------------------------------------------
\section{Recommended Reading}
\label{sec:recommended_reading}

The course does not follow a particular textbook; hence you are not required to have any. In case you would like to read more about some topics, we suggest the following references: \textit{Real Analysis with Economic Applications} by Efe Ok (henceforth Ok) and \textit{Math for Economists} by Simon and Blume (henceforth SB). In past years these two books have been used in Econ 2010. Additional readings: \textit{Principles of Mathematical Analysis} (aka baby Rudin) by Walter Rudin (henceforth, R) and \textit{Microeconomic Theory} by Mas-Collel, Whinston and Green (henceforth MWG).

% ---------------------------------------------------------------------
\section{Homework Assignments}
\label{sec:homework_assignments}

There will be a problem set every two lectures. The assignments will not be collected (though students are free to submit solutions and ask for feedback). The problem sets will consist of basic (and not-so-basic) questions regarding every topic we see. Their goal is to ensure you are aware of (and hopefully comfortable with) the math level that will be required throughout your first year, and to push you towards working with your peers for the first time.

% ---------------------------------------------------------------------
\section{Accommodations for Students with Disabilities}
\label{sec:accommodations_for_students_with_disabilities}

Brown University is committed to full inclusion of all students. Any student with a documented disability is welcome to contact me as early in the summer as possible so that we may arrange reasonable accommodations. As part of this process, please be in touch with Student and Employee Accessibility Services by calling 401-863-9588.

% ---------------------------------------------------------------------
\section{Tentative Schedule}
\label{sec:tentative_schedule}

\begin{tabularx}{\columnwidth}{llll}
  \toprule
  \multicolumn{2}{c}{Lecture}
  &
  \multicolumn{1}{c}{Topic}
  &
  \multicolumn{1}{c}{References}
                                                                                                                 \\\midrule
  \multirow{2}{*}{8/15 (Mon)} & 1st half & Introduction, Mathematical Proofs          & Lecture Notes         \\
                              & 2nd half & Intro to Topology, Limits                  & Ok A.1--A.3, B.1, C.1 \\ [6pt]
  \multirow{2}{*}{8/16 (Tue)} & 1st half & Sequences                                  & Ok A.1--A.3, C.1, C.5 \\
                              & 2nd half & Continuity                                 & Ok D.1, D.3           \\ [6pt]
  \multirow{2}{*}{8/17 (Wed)} & 1st half & Compactness, Extreme Value Thm             & Ok C.3--C.4, MWG M.I  \\ 
                              & 2nd half & Correspondences, Maximum Thm, Fixed Points & Ok E.1--E.3, MWG M.H  \\ [6pt]
  \multirow{2}{*}{8/18 (Thu)} & 1st half & Differentiation, Implicit Function Thm     & Ok K.1--K.2, MWG M.E  \\
                              & 2nd half & Unconstrained Optimization                 & SB 17, MWG M.D, M.J   \\ [6pt]
  \multirow{2}{*}{8/22 (Mon)} & 1st half & Constrained Optimization, Envelope Thm     & SB 18, MWG M.K        \\
                              & 2nd half & Integration                                & Ok A.4                \\ [6pt]
  \multirow{2}{*}{8/23 (Tue)} & 1st half & Linear Algebra                             & SB 10--11             \\
                              & 2nd half & Intro to ODE                               & SB 24--25             \\
  \bottomrule
\end{tabularx}

% ---------------------------------------------------------------------
\end{document}

% All-Student Orientation: August 31
% International Orientation: August 25-27
% Student of Color Orientation: August 29-30
